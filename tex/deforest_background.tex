%%%% BACKGROUND SECTION %%%%

In 1999 and 2000, the Government of India launched two major road
construction programs --- one aimed at upgrading several national
highway corridors and the other at connecting the remainder of India's
population to the road network. Together, these programs marked the
largest expansion of road infrastructure in Indian history and came at
a joint cost exceeding \$50 billion. This section provides background
information on both road construction programs.

\subsection{Rural Roads}

In 2000, the Indian government launched the Pradhan Mantri Gram Sadak
Yojana (PMGSY), or the Prime Minister's Village Roads Scheme. The
primary objective of the program was to provide new paved roads to
previously unconnected villages, although in practice this also
involved upgrading low quality roads in already connected villages. By
2015, over 400,000 kilometers of new roads were built, providing
access to the national road network to over 100 million rural people
in over 100,000 villages. Over 70\% of new rural roads were routes
that terminated in villages.

Rural road construction began toward the end of 2001 and was
continuing steadily through the end of the sample period in 2014 (See
Appendix Figure~\ref{fig:pmgsy_dates}). Villages were selected for
roads based on a set of guidelines issued by a national government
body, the National Rural Roads Development Authority. Notably, the
program prioritized construction of roads to larger villages;
district-level implementation plans were to first target all villages
with populations greater than 1000, followed by villages with
population greater than 500, and finally those with population greater
than 250.\footnote{Strictly speaking, the allocation was based on
  habitation population rather than village population. A habitation
  is a smaller unit of aggregation than the village; there are between
  one and three habitations in each village. In practice, habitation
  populations were pooled to the village level in many cases (see
  below). We aggregate to the village level because neither additional
  data nor maps are available at the habitation level.}

The rules were applied on a state-by-state basis, allowing states to
move from one threshold to another on their own timelines. In
practice, there were several other prioritization guidelines and
political patronage undoubtedly played a role, so that a village's
population relative to the threshold significantly influenced its
likelihood of receiving a road but was not definitive. For instance,
smaller villages could be connected if they were along the least-cost
path between larger prioritized villages, and proximate villages could
combine their populations to attain the eligibility thresholds. For
more details, see \citeasnoun{AN17} and \citeasnoun{NRRDA}.

\subsection{National Highways}

In 1999, the Indian government announced a plan to modernize its major
highways, the National Highways Development Project. The first
component of the project was the upgrading and widening of the Golden
Quadrilateral highway corridor (henceforth, GQ), so named because it
connected the four major cities in India: New Delhi, Mumbai, Chennai
and Kolkata. The second component was a similar upgrading of the the
North-South and East-West corridor (NS-EW), which would connect the
furthest corners of the country from Srinagar in the north to
Kanyakumari in the south, and from Porbandar in the west to Silchar in
the east. Panel A of Figure~\ref{fig:maps} shows both highway
corridors along with the major cities that were connected by them.

While the GQ and NS-EW projects were commissioned around the same
time, the government prioritized the implementation of the GQ and
construction of the NS-EW was substantially delayed. Construction on
the GQ began in 2001; 80\% was completed by 2004 and 95\% by 2006. In
contrast, by 2006 only 10\% of the NS-EW corridor was completed,
almost half of which was a set of highways which were shared with the
GQ \cite{GGK16}. By 2010, 72\% of the NS-EW was completed, and 90\%
was completed by 2015. The delay in the construction of the NS-EW
allows us to use the NS-EW corridor as a counterfactual for changes in
forest cover in the GQ corridor during and immediately following
substantial completion of the GQ.

Before these highways were widened and upgraded, the GQ and NS-EW
routes were already significant transportation corridors, but their
road quality and congestion were highly variable. The upgrading of
these networks dramatically improved their quality and reliability;
these were the first major long-distance divided highway networks to
be developed in India. The construction of the GQ changed national
supply networks and led to a substantial reallocation of manufacturing
firms into the GQ corridor \cite{SD12,GK16,GGK16}. The economic impact
of the NS-EW corridor has so far been little studied due to its later
completion date.
