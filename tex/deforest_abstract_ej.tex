There is a long-standing debate over whether new roads unavoidably
lead to environmental damage, especially forest loss, but causal
identification has been elusive. Using multiple causal identification
strategies, we study the construction of new rural roads to over
100,000 villages and the upgrading of 10,000 kilometers of national
highways in India. The new rural roads had precise zero effects on
local deforestation. In contrast, the highway upgrades caused
substantial forest loss, which appears to be driven by increased
timber demand along the transportation corridors. In terms of forests,
last mile connectivity had a negligible environmental cost, while
expansion of major corridors had important environmental impacts.
