
The development, maintenance and expansion of transportation
infrastructure is an important driver and correlate of economic
development around the world. In this paper, we provide causal
estimates of the ecological impact of two transportation investments
with global significance: India's massive expansion of rural roads and
its upgrading of national highways. Using identification strategies
established in the literature, we find that: (i) the new rural roads
had negligible effects on forest cover and; (ii) the highway
expansions had a large negative effect on forest cover, which may have
been driven by the expansion of wood-using
industries. Methodologically, we demonstrate the critical importance
of accounting for endogeneity and separately estimating the effects of
the construction period from the post-completion period.

Because of the different implementation and spatial structure of rural
roads and highways, it was necessary to use different empirical tests
to evaluate the environmental impacts of each. However, our
differential results are unlikely to be the result of having used
these different strategies. Our analysis of rural roads ruled out
even small effects on forest cover extending six years after the onset
of construction and four years after road completion---a time horizon
in which the highway expansions already had substantial impacts on
local forest cover. We also find no effects of rural roads on forests
in places with high local demand for wood from either individuals or
firms---which are the most likely drivers of forest change along
highway corridors. Finally, while most of our analysis of rural roads
focuses on local effects, we also found no effects at the wider
distance horizons at which forest loss responds to highway expansion.

Globally, road expansion is expected to dramatically increase through
the course of the 21st century. Some additional 25 million kilometers
of road infrastructure is projected to be built by 2050, a 60\%
increase over 2010 levels. Nine out of ten of these roads will be
built in developing countries \cite{WL14}. At the same time, tropical
forests in developing countries are increasingly under threat. These
forests not only provide global carbon benefits but also provide
important local ecosystems which support biodiversity as well as the
generally poor populations that rely on them \cite{BGM16}. Against the
background of this tension between economic development and
environmental conservation, understanding the relationship between
roads and forests is fundamental to a successful strategy for
sustainable development.

Crucially, we show that the impact of road construction depends on
what those roads connect.  The fiscal costs of the two large scale
transportation investments that we study were similar, but they had
vastly different environmental consequences.  Expansion of existing
highway corridors caused changes in the spatial distribution of
industry, which had dramatic effects on forest use in India.  In
contrast, building roads to connect smallholder farmers to new
markets had virtually no impact on local forests, even for those
farmers most likely to draw some part of their livelihoods from those
forests. 

