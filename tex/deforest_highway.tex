In this section, we aim to identify the causal impact of highways on
local forest cover. The identification challenge is that highways are
typically built to connect cities with current or anticipated economic
growth; if economic growth is correlated with forest cover changes for
any reason other than the direct effect of highways, then we cannot
interpret the correlation between highways and forests as a causal
effect.

We therefore focus on a set of places that happen to be in between the
targeted endpoints of India's new highways, as in \citeasnoun{GGK16}.
Both the Golden Quadrilateral (GQ) and the North-South and East-West
(NS-EW) corridors were upgraded with the objective of improving
connections between India's major cities and regions; the connection
of secondary cities and intermediate places on the route was a
secondary priority. Because these intermediate regions were targeted
incidentally rather than directly, the placement of the highways is
less likely to be driven by existing or anticipated economic growth.

We can further generate a plausible counterfactual that describes how
forest cover would have changed in the absence of the highway
upgrades. Like the GQ, the NS-EW route was an important transportation
corridor in 2000 and was to be upgraded before 2005 as part of NHDP, but
the project did not begin in earnest until several years after the GQ
was completed.  Our main estimates examine forest changes along the GQ
corridor during and after the construction years, as compared to
regions further from the GQ. We then test for effects along the NS-EW
route using a similar specification, showing there are no effects
along the second corridor until after 2008, as would be expected given
the construction delay.

As a starting point, Figure~\ref{fig:gq_means} plots kernel-smoothed
local regression estimates of mean forest cover and forest cover
change as a function of distance from each highway. Initial forest
cover (Panel A) is broadly similar across the two highways. Panel B
shows forest cover change from 2000-2008, also by distance to each
highway. Relative to the NS-EW (dashed line), forest cover within 100
km of the GQ (solid line) falls substantially between 2000 and
2008. At further distances the effects are similar across the two
highways, though there may be smaller relative gains for the GQ. We
present this as suggestive evidence of relative forest loss along the
GQ corridor during and after its construction. The rest of this
section generates formal tests for change, controlling for fixed
effects and other factors that may have simultaneously influenced
forest change.

\subsection{Highways: Empirical Specification}
\label{sec:highway_spec}

The simplest form of the difference-in-differences specification is
described by the following equation:
%
\begin{align} 
\label{eq:gq_simple} 
Forest_{ist} = \beta_0 + \beta_1 CLOSE_{is} + \beta_2 POST_{t}
+ \beta_3 CLOSE_{is} * POST_{t} + \epsilon_{ist}
\end{align}
%
In this specification, $i$ indexes a subdistrict in state $s$ and time
$t$, $CLOSE_{is}$ is an indicator for subdistricts close to the
highway, and $POST$ indicates years following the completion of the
highway. $Forest_{ist}$ is a measure of forest cover in subdistrict
$i$ and state $s$ at time $t$, usually log total forest cover. $\beta_3$
describes the differential change in forest between locations that are
near and far from the highway network after the highway is built,
controlling for the same geographic difference before the highway was
built. If new highways cause deforestation, we expect $\beta_3$ to be
less than zero. We conduct our analysis at the subdistrict level,
because subdistricts are contiguous regions that cover the whole of
India for which we can calculate a range of demographic and
socioeconomic controls. We weight results by subdistrict
area.\footnote{Results from a town- and village-level analysis with
  subdistrict clusters deliver nearly identical results. We could in
  principle conduct analysis at the grid cell level, but this would
  require imputation for control variables not available at the grid
  cell level.} There are approximately 4000 subdistricts in India.

We extend this simple specification in three ways. First, because we
do not have strong priors on which distances are near and which are
far, we use a flexible set of distance indicators to nonparametrically
identify highway effects at a range of distances. Estimates can still
be interpreted as the difference from a given band to the omitted
(most remote) distance band. This ensures that our result is not
dependent upon a particular definition of closeness. Second, because
the construction of India's national highways were multiyear projects,
we separate the $POST_t$ indicator into multiple periods to capture
construction and post-construction effects. Third, we add a wide set
of fixed effects and controls to improve precision and reduce bias
from omitted variables. The most flexible estimating equation is:
%
\begin{align} 
\label{eq:gq} 
  Forest_{ist} = \sum_{d =
  1}^D \sum_{t=2001}^{2014} \beta_{d,t} \mathbbm{1}
  (DIST_i \in (d^-,d^+), YEAR=t) + 
  \gamma_{st} + \boldsymbol X_i \cdot \boldsymbol \nu_{t} +
  \psi_d + \eta_{ist} \end{align}

\noindent The distance to the highway is divided into $D$ bands, the boundaries
of which are indexed by $d$. 
We include a distance band
fixed effect $\psi_d$, state-year fixed
effect $\gamma_{st}$ and a vector of subdistrict controls
($\boldsymbol X_i$) interacted with year fixed effects ($\boldsymbol
\nu_t$). The latter control for any differential time path of forest cover
that is correlated with baseline subdistrict characteristics.  Controls are
the same as in Equation~\ref{eq:rr1}. 
We include
locations up to a distance $D+E$ from the Golden Quadrilateral; the outer
boundary $(D,E)$ is the omitted distance category against which the other
estimates can be compared. Unless otherwise specified, we define $(D,E)$
as the 200-300km distance band.\footnote{Alternate choices of the
  range of the omitted group, including using the remainder of the
  country does not appreciably affect our estimates.}  $\beta_{d,t}$
identifies the change in forest cover from 2000 to year $t$, at
distance range $d$ from the highway, relative to the omitted distance
range $(D,E)$. The $\beta_{d,t}$ coefficients can thus be directly
interpreted as the effect of highway construction on forest cover
after $t$ years.  If new highways cause proximate forest cover loss,
we expect $\beta_{d,t}$ to take on negative values for low values of
$d$ in the periods $t$ after highway construction has begun. For
graphs, we include a set of indicator variables $\beta_{d,2000}$ which
describe baseline forest cover as a function of distance from the
highway.\footnote{We do not include subdistrict fixed effects because
  we want to generate coefficients on the distance band indicators for
  the omitted year 2000 --- these coefficients describe the baseline
  differences in forest cover between places that were near and far
  from the highway. However, inclusion of subdistrict fixed effects
  does not meaningfully change the results. We use state-year fixed
  effects rather than district-year fixed effects because we wish to
  test for meaningful effects of distance from highways that may
  extend beyond the radius of districts. District-year fixed effects
  would absorb true effects of the GQ that span distances larger than
  districts. The analysis of Goswami Ghani and Kerr \cite{GGK16} is
  entirely district level, giving us reason to expect meaningful
  cross-district effects. As expected, the inclusion of district-year
  fixed effects attenuates our results slightly but does not change
  the direction of effects nor eliminate statistical significance.}
Standard errors are clustered at the subdistrict level to account for
serial correlation. Because the regression above may have hundreds of
coefficients, we pool years or distances in different specifications
to improve interpretability.

We exclude areas within 200km of the nodal towns on the highway
routes, as we wish to identify effects on intermediate regions rather
than at the highway end points, as in Goswami Ghani and Kerr
\cite{GGK16}. Estimates of NS-EW treatment effects omit areas that are
within 200km of the GQ as they are plausibly being treated by the
other highway network. We do not omit NS-EW regions from the GQ
regressions because NS-EW construction has barely begun during the
periods of interest for the GQ analysis; however, regression results
are not changed by omitting places within 200km of NS-EW.

\subsection{Highways: Estimates on Forest Cover}
\label{sec:highway_results}

Panel A of Figure~\ref{fig:gq_dist} plots coefficient estimates from a
single estimation of Equation~\ref{eq:gq}, with distances from the
GQ highway divided into 10km bands, and years divided into a
single pre-construction year (2000), the construction period
(2001-2004), and two post construction periods (2005-2008 and
2009-2012). All estimates describe the difference between a given 10km
distance band from the GQ and the omitted category of
250-300km.\footnote{We include coefficients for the 200-250 km bands
  in order to plot treatment effects at these ranges. Effects in
  closer bands are very similar if we restrict the distance indicators
  to 200km and use 200-300km as the omitted group, because there are
  few differences across years in the 200-250km range.}  The solid
black line describes baseline forest cover as a function of distance
from the GQ corridor. The remaining lines show that forest cover
within 100 km of the GQ declines rapidly during the GQ construction
period and then continues to fall in the years following
construction. Effects are slightly smaller in the 100-150km bandwidth,
and statistically indistinguishable from zero at a distances greater than 150km from
the highway.

To alleviate the concern that these forest losses are explained by
existing trends in forest loss along existing highway corridors, we
run the same estimation for subdistricts along the NS-EW corridor and
show results in Panel B. As predicted, there are no differential changes
in forest cover close to the NS-EW route before 2008. Net forest
loss along NS-EW begins in the 2009-2012 period, and the distance effects then
look similar to those of the GQ.\footnote{Standard errors are omitted
  in the figure for visual clarity. For the GQ, differences between
  the 2000 estimates and the 2001-2004 estimates are statistically
  significant at the 1\% level for all estimates up to 150km. For the
  NS-EW, differences between the 2000 estimates and the 2009-2012
  treatment estimates are statistically distinguishable at the 1\%
  level until the 180km estimate.} This distance pattern of forest
cover loss is similar to that found in the Amazon \cite{PR07}. 
Effects along the NS-EW
corridor may be slightly smaller than along the GQ both because construction
took place slowly and was still incomplete at the end of the sample
period in 2014; the network
structure of highways mean that the value of any particular segment
depends on the completion status of other segments. Concentration of
industry along the GQ corridor may have also reduced the importance of
NS-EW as an intercity transportation corridor by the time the NS-EW
upgrades took place. 

 Table~\ref{tab:gq} presents regression estimates from
Equation~\ref{eq:gq}, with distances in 50km bands for legibility.
Each estimate describes the difference in forest cover between a given
distance band and the omitted category of 200-300km. Columns (1) and
(2) describe the impact of the GQ on forest cover.  The top four rows
of the table show estimates of construction period impacts on forest
cover at various distance bands. Places within 50km of the new highway
network lose 27 log points of forest cover (Column 1) or 1.3
percentage points of forest cover (Column 2, on a base of 7.5\%), and
the effects shrink at greater distances. The next four rows show
similar effects (still relative to year 2000) in the post-construction
period of 2005-2008. The final four rows show estimates of baseline
differences between the GQ and the regions further away; the level
differences are small relative to the treatment effects.

Columns 3 and 4 of Table~\ref{tab:gq} show comparable estimates along
the NS-EW corridor for the same time periods.  There are no detectable
changes in forest cover close to the NS-EW in the time period when
significant forest loss took place near the GQ. Along with
Figure~\ref{fig:gq_dist}, this should alleviate any concern that the
GQ treatment effects are driven by generalized deforestation along
existing highway corridors from 2001-2008.  These results are robust
to instrumenting for highway location using straight line instruments
connecting the nodal cities of the highway network, as employed by
\citeasnoun{GGK16}; we present analogous reduced form estimates to the
above in Appendix Table~\ref{tab:gq_iv}. These estimates alleviate the
concern that the particular routing of the GQ (but not the NS-EW) was
specifically targeted to places that may have already been losing
forest cover.

An alternate estimation approach would be to exploit the timing of
construction of each segment of the highway and to study forest cover
in the years before and after the nearest segment to a given location
was built. This would be analogous to the difference-in-differences
estimation used to study the impact of rural roads in
Section~\ref{sec:rural}. The results, which are consistent with the
findings above, are presented in Appendix
Table~\ref{tab:gq_timing}.\footnote{We did not use this as a primary
  specification because there are substantial network effects in
  highway construction. A region without an upgraded highway segment
  may experience greater transportation access if the regions around
  it experience upgrades. Equally, a nearby upgrade may not be of much
  value if it is an isolated upgrade to a not-yet-upgraded part of the
  corridor. We therefore would expect estimates from a specification
  exploiting timing to underestimate the full impacts of the highway
  construction, which is what we find in Appendix
  Table~\ref{tab:gq_timing}.} One useful feature of this analysis is
that it is directly analogous to our difference-in-differences
estimates of the impacts of rural roads (Equation~\ref{eq:rr1} and
Table~\ref{tab:panel}), making clear that the differential effects of
highways and rural roads are not an artifact of the different
empirical strategies used to evaluate them.

A final concern with these estimates of forest change is that they
could be describing displacement of forest loss from the hinterlands
to the highway corridors, or even net afforestation in the
hinterlands. This concern arises frequently in studies of
transportation projects with national scale, and is typically only
resolved by assumption through a structural modeling approach, which
is beyond the scope of this paper.  This said, large displacement
effects are made less plausible by the low quality of the broader road
network, and by the high transportation costs during the sample
period, weakening market connections with the hinterlands of these
highways.  While we cannot entirely rule out that there may be some
displacement effects, Panel B of Figure~\ref{fig:gq_means} suggests
that effects are driven by the highway corridor regions rather than
the hinterlands; forest cover diverges between the places close to the
GQ and NS-EW, but not in those far away.

\subsubsection{Mechanisms for Highway Effects}

We consider four possible mechanisms for the forest cover loss caused
by India's major highway networks: (i) increased demand for timber
products by firms due to local growth; (ii) increased demand for
firewood due to shifts in household fuel consumption; (iii) expansion
of agriculture into previously forested lands; and (iv) clearing of
trees for settlements and industry. This section presents
suggestive evidence that the deforestation along India's major
highways is predominantly caused by increased logging driven by local
timber demand.

To identify potential mechanisms, we use the regression specification
used to identify effects of highways on forest cover
(Equation~\ref{eq:gq}), with data from the economic and population
censuses which were undertaken in various periods between 1990 and
2013. Because the results above suggest forest cover changes within
100km of the highway network, we use 100km distance bands and define
200-300km as the omitted distance band.\footnote{We find virtually
  identical results when we use 50km distance bins.} The years in the
sample are determined by census availability.

We first look at changes in employment in the list of industries that
are directly downstream from timber harvesting (described in
Section~\ref{sec:data}). Panel A of Figure~\ref{fig:gq_mech} shows
results from a regression of log employment in major wood-consuming
sectors on the usual set of year-distance-band fixed effects. We graph
the point estimates on the 0-100km coefficient in each year that the
Economic Census is available (\textit{i.e.} the coefficients
$\beta_{0-100km,1990}$, $\beta_{0-100km,1998}$,
$\beta_{0-100km,2005}$, and $\beta_{0-100km,2013}$ from
Equation~\ref{eq:gq}). These estimates can be interpreted as the
difference in residual log employment between the 0-100km distance
band and the 200-300km distance band, after controlling for state-year
fixed effects and the controls described above. In 1990 and 1998
(before the GQ was begun), there is no significant difference between
areas close to the highway corridor and areas that are far from it,
nor is there a significant trend. By 2005, we see a 5\% increase in
employment in wood-consuming firms in the GQ corridor relative to the
hinterland, which continues to rise through 2013. Panel B shows
similar results for employment in logging firms; we omit 2005 because
logging firms were not distinguished from firms engaged in
afforestation in the 2005 Economic Census.  These two graphs suggest
that demand for wood from downstream firms is a plausible explanation
for local deforestation after construction of the
GQ.\footnote{Appendix Tables~\ref{tab:app_gq_mech_ec} and
  \ref{tab:app_gq_mech_land_fuel} show complete regression results for
  all mechanism tests. As would be expected, employment in logging
  firms is more geographically diffuse, as those firms reach further
  into the GQ's hinterland.} Logging firms were more common
along the transport corridor even before the GQ was built, but there
is no suggestion of a pretrend that could explain what we see after
highway construction. Note that employment in other sectors of the
economy, which also consume wood, exhibit similar treatment effects
close to the new highways \cite{GGK16}.

Panels C through E of Figure~\ref{fig:gq_mech} show the effects of the
GQ upgrades on household fuel consumption, for which data are
available in 2001 and 2011. We observe marginal increases in firewood
and imported wood use, and comparable reductions in use of local
non-wood fuels. These effects are not statistically significantly
different from zero, nor are they large enough to explain a 20\%
reduction in forest cover in the neighborhood of the GQ
corridor. Panel F shows that land use shifts slightly \textit{away}
from agricultural uses following the construction of the GQ, breaking
a previous upward trend, making agricultural extensification an
unlikely explanation for the treatment effects.

It is difficult to directly test the last hypothesis that net forest
loss has come from the expansion of land dedicated to settlement and
industry, because data on land dedicated to settlement and industry
only becomes available in the 2011 Population Census. However, it is
implausible that settlement and industrial expansion could explain a
20\% reduction in forest cover in a distance band as wide as 100km
around a 6000km long highway corridor. In 2011, only 6.7\% of of rural
land was used for settlement and industry.

The large relative forest losses along the GQ corridor are unlikely to
be direct effects of the road construction process. Although we found
evidence of these effects in the construction of rural roads in
Section~\ref{sec:rural}, these were very local, temporary, and an
order of magnitude lower in size. It is also implausible that direct
construction effects would extend more than a few kilometers from the
roadway.

In conclusion, we find evidence that expansion of industry
demand for timber can explain forest loss in the GQ corridor, and we
can rule out agricultural expansion, changes in household fuel
consumption and settlement expansion as mechanisms.
