Does human economic progress have an unavoidable environmental cost?
This is a central question for policymakers pursuing sustainable
development and has been a long-standing debate in both the
conservation and the economics literature
\cite{AD95,GK95,SCB96,AL01,FR03,D07,AMSW13}. A key pillar of economic
development is large-scale investment in transportation infrastructure
that reduces the costs of moving goods and people across
space. Concern has been expressed about the potential environmental
cost of such investments, and of increased trade more generally
\cite{CT94,ACT01,CT04,FR05}, but researchers have struggled to
identify causal estimates of the impact of transportation
infrastructure on local environmental quality.

The most omnipresent of transportation investments are roads. We focus
on the impact of road construction and expansion on forest loss as it
is among the primary environmental concerns associated with new road
construction.  Forest cover loss is globally and locally important,
generating global greenhouse emissions \cite{IPCC,JS17} and local
health externalities \cite{BBPS15,TG17}. Analysis by the IPCC suggests
that restoring and protecting forests could yield almost a sixth of
the emissions mitigation required to prevent runaway climate change by
2030 \cite{IPCC18}.

Because of the high cost and high expected return of roads, their
placement typically depends on various economic and political factors,
making causal identification of their impacts difficult.  For example,
new roads may be targeted to regions with expanding agricultural land
use; these roads may be a response to activities that are already
causing forest cover reduction, making it difficult to isolate the
direct impact of the roads.  While many earlier studies have
documented changes in forest cover following the construction of new
roads, none have addressed the endogeneity of road placement beyond
the inclusion of control variables and in a few cases, location fixed
effects. Further, most of these studies have focused on large highways
built into the Amazon rainforest \cite{AP99,PR07,WR08}; while these
highways are important in terms of potential deforestation, their
impacts are of uncertain relevance for the set of potential rural
roads and highways that policy-makers in developing countries are
considering today. The majority of road projects in the decades ahead
are likely to be last-mile roads to people not currently connected to
the road network and upgrades of existing transportation corridors
into modern highways.

In this paper, we take advantage of a validated satellite-based
measure of forest cover (Vegetation Continuous Fields or VCF), which
makes it possible to study the impacts of two large-scale
transportation projects in India.  The first of these was an
initiative to upgrade two major transportation corridors: the 6000 km
``Golden Quadrilateral'' network (GQ) connecting the country's four
largest cities, and the comparably-sized ``North-South and East-West''
network (NS-EW) connecting the country's four cardinal endpoints in a
cross. Both corridors were already used for cross-city transportation
before 2000, but over the following fifteen years they were upgraded
into world class divided highways.  The second project was a rural
road construction program, under which over 100,000 new paved rural
feeder roads were built, ten kilometers in length on average,
providing new connections to over 100 million rural residents. Each
project has exceeded ten billion dollars in cost to date and has
caused a significant reallocation of local economic activity
\cite{AN17,GGK16}.

Theoretically, the effect of road investments on local forest cover can be
positive or negative.  New roads can increase forest cover loss by:
(i) providing external markets for forest resources, especially timber
and firewood; (ii) providing external markets for agricultural
products, motivating extensification of agriculture into forested
land; and (iii) increasing the value of land for settlement and
industry, resulting in forest clearing.  On the other hand, paved
roads could also reduce forest cover loss by (i) improving local
household and industry access to substitutes for local forest
resources, especially firewood; (ii) providing access to external
output and labor markets, lowering the relative returns to clearing
forests for agricultural land as well as to harvesting other forest
products such as firewood. Given the substantially different nature of
rural feeder roads and national highways, we can also expect the
importance of any of these channels to vary by the type of
road. 

To evaluate the impact of rural roads, we first use a regression
discontinuity approach, exploiting an implementation rule that
discontinuously raised the probability of road construction in
villages with population above an arbitrary threshold.  Second, we use
a difference-in-differences specification that exploits the exact
timing of road construction. Both approaches show zero effects of new
roads on forest cover. The estimates are precise; we can rule out
gains larger than 0.6\% and losses greater than 0.2\% in forest cover
up to five years after roads are completed. Further, we find zero
effects for sample subgroups where we might expect losses to be
greater, such as villages with greater baseline forest cover or with
very poor or forest-dependent residents. We also find zero change in
household firewood use in treated villages. We do identify marginal
(0.5\%) reductions in forest cover during the road construction
period; these reductions are reversed soon after roads are completed,
but there is no evidence that forest cover continues to rise. We show
that ignoring these construction period effects could lead to biased
impact estimates. These roads have no effect on forest cover in spite
of significantly altering economic opportunities for people in
villages \cite{AN17,AAN17}.

Causal identification for impacts of highways is much more difficult
than for rural roads, because in almost all cases, new highways are
small in number and are built along existing transportation
corridors. We take the approach of comparing changes in forest cover
in areas that are near and that are far from the new highways. While
we do not have data covering the period before the construction of the
Golden Quadrilateral, the North-South/East-West highway route provides
a plausible counterfactual, in that it is a highway of comparable size
and importance that was announced simultaneously and on a similar
construction schedule to the GQ, but its construction was pushed back by
approximately eight years due to bureaucratic
delays. \citeasnoun{GGK16} take a similar approach in comparing these
two networks to study the impacts of the GQ on manufacturing
activity.\footnote{On the impacts of the Golden Quadrilateral on firms
  in India, see also \citeasnoun{SD12} and \citeasnoun{GK16}.}

In sharp contrast to rural roads, we find that the highway upgrades
have had substantial negative effects on forest cover.  Following
construction of the GQ, we find a 20\% decline in forest cover in a
100 kilometer band around the highway, an effect that persists for at
least eight years. We find no change in forest cover along the NS-EW
corridor until construction accelerates in 2008, at which point we
also observe local forest cover loss. The timing of relative forest
loss around the construction of each corridor supports a causal
interpretation of these estimates.  Because forest cover in India is
rising on average during the sample period, these are net effects on
forest cover, combining increases in deforestation and reductions in
afforestation.

These highways appear to have depleted forest cover by increasing
timber demand in their vicinity, which has wide ranging effects into
the hinterlands of the transport corridors. Following the construction
of the GQ, we find a substantial upward trend break in employment in
proximate firms that use timber and wood as primary inputs, as well as
employment in logging firms. Additional tests reject the competing
mechanisms; there are no increases in agricultural land use or changes
in local firewood consumption along the highway corridor.

This paper makes two central contributions. First, we generate the
first causal estimates of the impact of large scale transportation
infrastructure investments on natural resource
depletion.\footnote{Many studies describe cross-sectional
  relationships between roads and forest cover or forest loss
  \cite{CD96,AK99,AP99,CPGGB01,GL02,DHURG11,BCCL14,LDU15,DW16}. A
  small number of studies examine forest loss in areas with new roads
  but do not address the endogeneity of road placement
  \cite{PR07,WR08}. The closest study to ours is ongoing work by
  \citeasnoun{K17}, who uses a difference-in-differences design
  similar to our first strategy (but does not look at highways),
  finding that India's new rural roads marginally increased forest
  cover. The differences may arise because \citeasnoun{K17} does not
  distinguish between construction and post-construction periods, and
  includes villages that never receive roads as part of the control
  group. We show in Section~\ref{sec:rural} that both of these choices
  may lead to biased treatment effects.}  In so doing, we contribute
to a long literature on the trade-offs and synergies between economic
development and environmental conservation.\footnote{On the general
  relationship between economic development and the environment, see
  \citeasnoun{DV94}, \citeasnoun{AD95}, \citeasnoun{GK95},
  \citeasnoun{SCB96}, \citeasnoun{AL01}, \citeasnoun{DLWW02},
  \citeasnoun{FR03} and \citeasnoun{DS04}. On deforestation
  specifically, see \citeasnoun{KT99}, \citeasnoun{BO12},
  \citeasnoun{AMSW13}, and \citeasnoun{JS17}. \citeasnoun{ALMS17}
  provide causal evidence that rural \textit{electrification}
  mitigated forest loss in Brazil. For an exhaustive review on drivers
  of deforestation, see \citeasnoun{FB14}. For a literature review on
  impacts of highways and rural roads on outcomes other than the
  environment, see \citeasnoun{AN17}.}

Second, this is the first paper to show that the impact of roads on
deforestation is a function of which markets are being connected by
those roads. Last-mile rural roads provide connectivity to small local
markets, facilitating exits from agriculture but without significantly
changing industry's access to forest products \cite{AN17}. In
contrast, highways dramatically change the geographic distribution of
industry \cite{GGK16}; in India at least, this appears to have
substantial environmental consequences.

Our estimates are particularly relevant as the infrastructure agenda
in Sub-Saharan Africa and South and Southeast Asia is likely to
prioritize exactly the kinds of infrastructure investments that we
study here --- new feeder roads and expansion of existing corridors
--- as opposed to the large highways through virgin rainforest that
have been the subject of much of the earlier work on roads and
deforestation. China's Belt and Road Initiative is the signature
example, which aims to promote the construction of large scale highway
corridors across Southeast and Central Asia, most of which are
expansions of existing roadways \cite{HT18}. In sub-Saharan Africa,
where only 30\% of rural people live within two kilometers of a road, last mile
access is a major policy priority \cite{WB06}.

Finally, we raise an important methodological issue in the literature
on estimating impacts of infrastructure. Large-scale infrastructure
often takes many years to build and involves significant land clearing
and economic activity during the construction process. In both our
examination of highways and of rural roads, we find that forest loss
begins during the construction period; in either case, estimates based
strictly on the timing of infrastructure completion would
underestimate the environmental impact of roads.

The rest of the paper is organized as follows. The next
section describes India's rural road and highway construction
programs. In Section~\ref{sec:data}, we describe the data on
forest cover and roads, as well as other secondary datasets used in
our analysis. Section~\ref{sec:rural} presents empirical strategy
and results describing the impact of rural roads on deforestation. 
Section~\ref{sec:highway} presents the empirical strategy and impacts
of highway expansions, and Section~\ref{sec:conclusion} concludes.
