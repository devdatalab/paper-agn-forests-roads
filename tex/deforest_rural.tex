This section describes the impact of new feeder roads on local
deforestation. The main challenge to causal identification of the
impacts of rural roads is endogeneity.  Because roads are costly to
build, their placement is typically correlated with other factors that
could also be predictors of deforestation. For example, roads could be
targeted to places that are expected to grow or to places that are
lagging economically. Road placement may also depend on geographic
(e.g. slope, terrain, soil quality) or political factors.  Any of
these scenarios would bias OLS estimates of the effect of new roads on
deforestation.\footnote{Appendix Table~\ref{tab:ols} shows estimates
  from cross-sectional OLS regressions of village-level log forest
  cover in 2001 on an indicator variable that takes the value one if a
  village has a paved road in 2001. While the bivariate relationship
  is strongly negative and highly statistically significant, the
  estimate gets progressively closer to zero as we add village-level
  controls and fixed effects, implying substantial selection on
  observables in the presence of roads. Selection on unobservables is
  plausibly also important, making the OLS estimates unreliable for
  causal inference.}  Causal identification of the impact of new roads
therefore relies on some kind of variation in road placement or timing
that is plausibly exogenous. To study the impact of rural roads, we
rely on (i) an implementation rule that led to a discontinuity in the
probability of a village getting a new road based on arbitrary
population cutoffs; and (ii) variation in the specific year that a
targeted village was treated. We focus our analysis on forest cover in
the vicinity of connected villages. Because newly connected rural
villages are mostly small and isolated, and because most of the new
roads terminate in villages rather than providing new long distance
corridors, these roads are unlikely to have had important general
equilibrium effects on more distant areas.

\subsection{Rural Roads: Regression Discontinuity Specification}

We begin by exploiting the eligibility rule that prioritized villages
for new roads based on arbitrary population thresholds. Given the
imperfect compliance with these eligibility rules (described in
Section~\ref{sec:bg}), we employ a fuzzy regression discontinuity (RD)
design. 
We limit the RD analysis to
states in which administrators adhered closely to
population threshold rules.\footnote{We identified these states with
  the help of officials at NRRDA. They include Chhattisgarh, Gujarat,
  Madhya Pradesh, Maharashtra, Orissa and Rajasthan. The
  difference-in-differences analysis below uses all states that built
  any roads in the sample period.} 

We use an optimal bandwidth local linear regression discontinuity
specification \cite{IL08,IK12,GI14} to identify the change in forest
cover caused by a new road at the treatment threshold. We use the
following two stage least squares specification:
\begin{align} 
\label{eq:rd1} 
\nonumber Treatment_{vds} &= \gamma_0 + \gamma_1\cdot \mathbbm{1} (pop_{vds} \ge T_s) + \gamma_2 (pop_{vds} - T_s)
               + \gamma_3 (pop_{vds} - T_s)\cdot \mathbbm{1} (pop_{vds}
               \ge T_s) \\  &+ \nu_d + \boldsymbol \theta \boldsymbol{X}_{vds}
               + \epsilon_{vds} 
\\
\nonumber Forest_{vds} &= \beta_0 + \beta_1\cdot Treatment_{vds} + \beta_2 (pop_{vds} - T_s)
             + \beta_3 (pop_{vds} - T_s)\cdot \mathbbm{1} (pop_{vds} \ge T_s) 
\\  &+ \mu_d + \boldsymbol \kappa \boldsymbol{X}_{vds} + \eta_{vds}
\end{align} 
\noindent 
$Forest_{vds}$ is forest cover in village $v$, district $d$ and state
$s$, and $Treatment_{vds}$ is an indicator equal to one if a new road
was built in village $v$. $pop_{vds}$ is the population of village $v$
and $T_s$ is the treatment threshold used in state $s$.\footnote{The
  treatment threshold varies with state because some states used a
  threshold of 500 and others were using a threshold of 1000. States
  used the lower treatment threshold when they had few villages with
  population over 1000 that did not already have roads. Officials at
  the National Rural Roads Development Agency provided us with
  information on which states were using which cutoffs, which we then
  verified in the data. Madhya Pradesh used both the 500 and 1000
  treatment thresholds for roads built in the same period; we include
  separate fixed effects for the set of villages in the neighborhood
  of each threshold. Because the optimal regression discontinuity
  bandwidth is close to 100, there is no overlapping between these two
  groups. Few villages around the lowest population threshold of 250
  received roads so we do not use this threshold for analysis.}
$\mu_d$ and $\nu_d$ are district fixed effects; we find virtually
identical results with fixed effects at higher or lower geographic
scales. $\boldsymbol{X}_{vds}$ is a control for baseline forest cover;
like the fixed effects, the control is unnecessary for identification
but improves precision. This is a cross-sectional regression where
$\beta_1$ identifies the effect of new roads on forest cover in a
given year. Outcomes are measured in the final year in the sample
data, which is 2013.\footnote{We find similar results if we pool
  outcome years from 2010 through 2013 and cluster standard errors at
  the village level (not shown). Standard errors are slightly smaller
  with this alternative approach, at the cost of putting more weight
  on roads which have been built for shorter periods of time.}

Appendix Figure~\ref{fig:rd_balance} shows regression discontinuity
balance tests for a set of variables measured in the baseline period;
Appendix Table~\ref{tab:rd_balance} presents the regression estimates on these
tests using Equation~\ref{eq:rd1}. None of the regression
discontinuity estimates are significantly different from zero at
baseline. Appendix Figure~\ref{fig:rd_mccrary} shows that the density
of the running variable is continuous around the treatment threshold
\cite{MC08}.

\subsection{Rural Roads: Regression Discontinuity Results}

Figure~\ref{fig:rd} shows a graphical representation of the regression
discontinuity estimates of the impact of rural roads on forest
cover. Panel A shows the first stage; the Y axis shows the share of
sample villages that received new roads by 2013 under PMGSY as a
function of their population relative to the treatment
threshold. Villages above the threshold are about 16 percentage points more likely to
receive new roads and the discontinuity is evident. Panel B shows the
first stage estimate separately for each outcome year; each point in
the figure represents the $\gamma_1$ coefficient from
Equation~\ref{eq:rd1}, where the dependent variable takes the value
one if a village received a new road by the year indicated on the X
axis. We can see that roads built before 2007 were not prioritized
according to the population threshold rule; the first stage of the RD
becomes noticeable after 2008 and continues to rise until 2014.

Panel C of Figure~\ref{fig:rd} plots village-level log forest cover in
2013 against the population relative to the treatment threshold, in
population bins. If roads significantly affected local forest cover,
we would expect to see a discontinuity at the treatment threshold
analogous to that in Panel A; no such treatment effect is
evident. Panel D shows the reduced form treatment effect of
above-threshold population on forest cover ($\beta_1$) in each year
separately; as in Panel B, each point is an estimate from a separate
regression, where the dependent variable is the log of forest cover
for the year on the X axis. If the new rural roads significantly
affected forest cover, we would expect to see a change in the
coefficient following 2008 when administrators began to adhere to the
population implementation rule. Instead, the effect is very close to
zero both before and after 2008, indicating that new rural roads had
negligible effects on forest cover.

Table~\ref{tab:rd} shows analogous regression estimates, where the
dependent variable is forest cover as measured in 2013.  Column 1
shows the first stage estimate of a 16 percentage point increase in
the probability of road treatment for villages just above the
eligibility threshold. Columns 2 and 3 confirm there is no reduced
form effect on either log or average forest cover. Columns 4 through 6
test for treatment effects in villages that might be expected to
respond more to new roads. These are: villages with above-median
baseline forest cover (Column 4); villages with above-median
population shares of constitutionally described ``backward"
communities (Scheduled Tribes) who often derive livelihoods from
forests (Column 5); and villages with below median assets, who might
depend more on forests for fuelwood (Column 6). There is no evidence
of impacts of roads in any of these groups.\footnote{Appendix
  Table~\ref{tab:rd_het} shows further that roads do not significantly
  affect forest cover in villages defined by high or low town
  distance, market access, nor in villages in subdistricts with above
  median employment in the logging sector or in industries that are
  heavy consumers of wood.}  Columns 7 and 8 show IV estimates on log
and average forest cover. The IV estimates respectively rule out a
0.14 gain and a 0.11 loss in log forest cover with 95\% confidence, or
approximately a one percentage point change in average forest
cover. The average treated village in the sample received a new road
in 2008, so these estimates reflect cumulative forest change five
years after a village is connected.  Results are robust to different
controls or fixed effects and different bandwidth
choices.\footnote{Results at many different bandwidths are shown in
  Appendix Table~\ref{tab:rd_band}.} Appendix Table~\ref{tab:rd_fuel}
uses the RD specification to show further that there are no changes in
household fuel use following completion of a new road.

\subsection{Rural Roads: Difference-in-Differences Specification}

The regression discontinuity design estimates causal impacts of roads
under minimal assumptions, but is limited to estimating a LATE in the
neighborhood of the treatment threshold in states that closely
followed implementation rules on population thresholds. We can make
greater use of our data and obtain tighter treatment estimates using a
difference-in-differences specification that exploits the differential
timing of road treatment in each village. For this empirical test, we
limit the sample of villages to those that received a road at
\textit{some} point during the road construction program, and use
outcomes in later-treated villages as a control group for villages
that were treated earlier.  We specifically estimate the following
equation:
%
\begin{align} 
\label{eq:rr1} 
Forest_{vdt} = \beta_1 \cdot Award_{vdt} +  \beta_2\cdot
Complete_{vdt} + \boldsymbol \alpha_v + \boldsymbol \gamma_{dt} + \boldsymbol
X_v \cdot \boldsymbol \nu_{t} + \eta_{vdt} 
\end{align}
%
\noindent
$Forest_{vdt}$ is a measure of forest cover in village $v$ and
district $d$ in year $t$. $Award_{vdt}$ is an indicator that takes the
value one for the years where a contract has been awarded for the
construction of a road to village $v$ but the road construction is not
yet complete. $Complete_{vdt}$ is an indicator that takes the value
one for all years following the completion of a new road to village
$v$. We separate these two periods because the road construction
process may have effects on forest cover (such as clearing of forested
area to make room for the physical placement of roads) that are
theoretically distinct from the economic effects of a village having a
new road. Village fixed effects ($\boldsymbol \alpha_v$) control for
all village-level time-invariant unobservables, while district-year
fixed effects ($\boldsymbol \gamma_{dt}$) control for any pattern of
regional shocks.\footnote{Results are unchanged by replacing these
  with state-year or subdistrict-year fixed effects.}  We also
interact a vector of baseline village controls $\boldsymbol X_v$
(baseline forest cover, village population and distance from the
village to the nearest towns) with year fixed effects. These control
for any differential time path of forest cover that is correlated with
baseline village characteristics. These controls are particularly
important because larger villages are more likely to be treated
earlier due to program implementation rules.  Standard errors are
clustered at the village level to account for serial correlation.

We can interpret $\beta_1$ and $\beta_2$ as the effects of road
construction activities and the effects of new roads, respectively;
both coefficients describe outcomes relative to the period before any
construction began. We restrict our sample from the universe of
villages in India to those that had no road in 2000 and had a road
completed during the study period. We do this so as not to compare
villages that received new roads with those that did not; the
endogeneity problem in such a comparison is severe.\footnote{As we
  show above, a minority of roads were allocated strictly due to the
  village population thresholds. There are enough of these to estimate
  a regression discontinuity test on local compliers, but not enough
  to assume that all treated villages are selected as good as
  randomly.}  Identification rests on the assumption that, among the
set of villages that received roads in the sample period, there are no
other systematic changes specific to villages in the years that roads
were awarded and completed that are not caused by the roads
themselves.

\subsection{Rural Roads: Difference-in-Differences Results }
\label{sec:panel}

The difference-in-difference estimates of the impact of rural roads on
village-level forest cover are summarized by Figure~\ref{fig:panel}.
These graphs show the residual of log forest cover --- after taking out
fixed effects and controls described above --- as a function of the
number of years elapsed since a road was completed in a given
village. Panel A shows all previously-unconnected villages that
received new roads between 2001 and 2014.  Panel B restricts the set
of villages to those with above median forest cover in 2000.  We show
only four years before and after road construction because wider
windows have more variable sample composition across estimates; this
occurs because we observe different length of pre- and post-periods
for different villages depending on their date of
treatment.\footnote{Appendix Figure~\ref{fig:panel_4} shows a wider
  time window around treatment; the pattern is the same.} Two patterns
are evident in the figure.  First, there is a statistically
significant reduction in forest cover approximately two years before
road construction is complete. Second, forest cover marginally
increases in the four years after road completion recovering some or
all of the pre-treatment drop.

Given that these rural roads took one to two years to build, this
pattern is consistent with a small degree of forest loss
(approximately 0.5\%) during the road construction period, with
partial or complete recovery afterward. We test this directly in
Table~\ref{tab:panel}, which shows estimates from
Equation~\ref{eq:rr1}. Our main estimate in Column 1 shows that
villages lose 0.5\% of their forest cover during the period between
the awarding of a road construction contract and the completion of a
road. However, that forest loss is fully restored in the period after
the road has been completed; the estimate of 0.002 log points on the
completion indicator can be interpreted as the difference in forest
cover between the post-road and the pre-award periods. Relative to the
pre-award period, we can rule out gains larger than 0.6\% and declines
larger than 0.2\% in forest cover. In Column 2, we show that failing
to account for the award period would lead to the estimation of a
marginal forest cover gain of 0.5\% because it would incorrectly
attribute the construction period loss to the pretrend. This result
highlights the importance of accounting for the construction period
when studying the environmental impacts of new infrastructure. Columns
3 and 4 present estimates where forest cover is measured as the
average share of each pixel that is covered by forest; results are
similar.  These estimates are based on different lengths of
post-construction periods in different villages, but on average they
show effects for four years after treatment.\footnote{Appendix
  Table~\ref{tab:panel_robust} shows that these estimates are robust
  to a range of specifications including the use of village time
  trends, subdistrict-year fixed effects (instead of district-year
  fixed effects) and using a limited sample of roads for which we have
  at least 4 (or 5) years of both pre-treatment and post-treatment
  data. Appendix Table~\ref{tab:mis_ests} shows additional
  specifications. Column 1 adds villages that did not receive roads in
  the sample period, the specification used in \citeasnoun{K17}. Like
  \citeasnoun{K17}, we find a positive treatment coefficient; however,
  Column 2 shows that this is not robust to the inclusion of
  village-specific time trends, indicating that never-treated villages
  are on different forest cover trends from treated villages. Columns 3
  and 4 show that our main estimate is robust to village-specific time
  trends. Column 5 and 6 define the treated area as a circle around
  the village with a radius of 5km and 50km, respectively; as in the main
  specification, we find no treatment effects at these radii.}

Table~\ref{tab:panel_het} shows these estimates along the same
dimensions of heterogeneity described above. Effects are broadly
similar whether we cut the sample on baseline forest cover, population
share of Scheduled Tribes, or asset poverty. There is thus no evidence
that our zero results are hiding differential positive and negative
effects in different places.\footnote{Equally, we find no effects of
  roads on forest cover when splitting the sample on distance to the
  nearest town or on market access, nor in subdistricts with above
  median employment in logging or in industries with high consumption
  of wood (Appendix Table~\ref{tab:panel_het_app}).} It is also
unlikely that outmigration of individuals following road construction
is significantly biasing our findings; these rural roads are not
associated with significant population change at the village level
\cite{AN17}. Rural-to-urban migration has also been much slower in
India over the sample period than in other countries at comparable
levels of income.

The panel estimates confirm the finding in the regression
discontinuity analysis, using a different set of villages with a
different local average treatment effect; the evidence is clear that
new rural roads have had a negligible effect on local forest cover.

