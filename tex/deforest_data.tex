\label{sec:data}

%%% DATA SECTION %%%

To estimate the effects of new roads on forest cover, we combine five
different national data sources. We use a validated high resolution
satellite-based measure of forest cover. Data on rural roads come from
the administrative implementation data generated by the rural road
construction program, and geographic data on new major highway
networks come from national highway maps. While these datasets form
the basis of our core specifications, we also use data from the 1991,
2001 and 2011 Population Censuses and 3rd through 6th rounds of the
Economic Census to control for location characteristics and explore
mechanisms of treatment effects. All of these are census datasets that
describe the entire population of India and are geocoded to the
village, town and subdistrict levels. This section describes the
details of how we prepare and combine all of these datasets.
Table~\ref{tab:summary} shows summary statistics for all variables used.

\subsection{Forest Cover}

Detailed and reliable administrative records on forest cover and
deforestation rarely exist, especially in developing
countries. Instead, we obtain high resolution time series estimates of
forest cover using a standardized publicly-available satellite-based
dataset. Vegetation Continuous Fields (VCF) is available at 250m
resolution and provides annual tree cover from 2000--2014 in the form
of the percentage of each pixel under forest cover \cite{THCDS11}. For
our primary specification, we define forest cover as the total log
pixel value plus one in a given geographic area.\footnote{Results are
  robust to using the inverse hyperbolic sine transformation instead
  of log plus one.} Results are robust to using the average percentage
of forest cover in each village.

The VCF measure is a prediction of the percentage of a pixel that is
covered by forest, generated from a machine learning model based on a
combination of images from MODIS and samples from
higher resolution satellites. The measure employs not only the visible
bandwidth but also other bandwidths. For example, VCF uses thermal
signatures because forested areas tend to be cooler than non-forested
plantation areas, allowing VCF to (partially) distinguish between
forest cover and plantations. To the extent that thermal signatures
and other correlates can distinguish forests from non-forest
plantations, VCF substantially improves upon the Normalized
Differenced Vegetation Index (NDVI) that has been widely used in
understanding the causes of deforestation (for example,
\citeasnoun{FR03}). For all analyses, we restrict the sample of
villages to those that had non-zero forest cover in 2000, a year
predating the construction of all roads considered in this
research. This is also the earliest year that these forest cover
datasets are available.\footnote{Fewer than 10\% of villages have zero
  forest cover in 2000; 95\% of these villages have less than 1\%
  forest cover in 2014; the mean of forest cover for pixels with
  non-zero forest is 12.76\% in 2000.}

Some earlier studies have used the Global Forest Cover (GFC) dataset,
which describes baseline forest cover in the year 2000, and a binary
indicator for the year of deforestation for each 30mx30m pixel.  In
the GFC data, a pixel is considered deforested if over 90\% of 2000
forest was lost by a given year, or reforested if a pixel goes from
zero forest in 2000 to positive forest cover by 2012 \cite{H13}. While
GFC and VCF are both based on satellite imagery, GFC is less useful
for the study of forest cover in India, because forest change in India
is not well summarized by a binary deforestation indicator. The VCF
measures suggest that forest cover rose 15\% over the sample period,
an estimate consistent with official and international
sources. Because most of these gains are in areas that had some
pre-existing forest, they are not recorded by GFC. GFC also does not
describe partial forest loss, while VCF does. We can replicate GFC
estimates by restricting the VCF data to forest losses, but they miss
a significant share of forest change in the sample period. Because
92\% of villages are larger than the VCF cell size, the resolution
advantage of GFC would be minimal. In the cross-section data from the
year 2000, VCF and GFC have a correlation coefficient of 0.92 with
each other, as compared to respective correlation coefficients of 0.71
and 0.67 with an NDVI measure based on the choices of
\citeasnoun{FR03}. Appendix Figure~\ref{fig:gfc_ndvi} presents heat
maps of forest cover in 2000 according to these three datasets, which
convey clearly the similarity between VCF and GFC, and the difference
of both of these from NDVI.

We matched forest cover data to the 2011 Population Census village, town
and subdistrict boundaries using geographic boundary data purchased
from ML InfoMap. In remote parts of India, we received only settlement
centroids rather than village boundaries. We generated Thiessen
polygons for these villages; all results are robust to excluding this
set of villages. Panel B of Figure~\ref{fig:maps} shows a heat map of
baseline forest cover in India. While contiguous areas of very dense
forest are geographically concentrated, areas with 20-40\% of their
land covered by forest are found throughout the country.

\subsection{Rural Roads}

We scraped village-level administrative data describing the
construction of rural roads from the program's online management
portal.\footnote{The data is publicly available at
  http://omms.nic.in.} For each road, the data provide the names of connected
villages, the date when the contract for road construction was
awarded, and the date of road completion. While data were reported at
the sub-village (habitation) level, we aggregated the data to the
village level to match our other data sources. We define a village as
treated if any habitation in the village was provided with a new road.
The data construction and scraping approach is described in detail in
\citeasnoun{AN17}. The dataset describes over 100,000 new roads built
between 2001 and 2014; we limit our sample to areas with non-zero
forest cover and no paved road in the baseline year, leaving
approximately 65,000 new roads in the analysis
sample.\footnote{Results are robust to including upgrades and/or villages
  with no forest cover at baseline. These would be expected to
  attenuate non-zero treatment effects, thus their exclusion if
  anything biases us against finding zero effects.}

\subsection{Highways}

Construction dates and geocoordinates for the Golden Quadrilateral and
North-South and East-West corridors were generously shared with us by
\citeasnoun{GGK16}. We linked these to the village, town and
subdistrict polygons described above by calculating straight line
distances from polygon centroids to the nearest point on each highway.

\subsection{Population and Economic Censuses}

We matched all villages and towns from the 1991, 2001 and 2011
population censuses using a combination of incomplete keys provided by
the Registrar General and a set of fuzzy matching algorithms based on
village and town names. The population censuses describe village and
town public goods, village amenities (such as schools and medical
centers) and household characteristics, including the primary source
of cooking fuel. Fuel use is reported as the share of households in a
location using firewood (68\% of households at baseline), imported
fuels (chiefly propane, 8\%) or local nonwood fuels (crop residue and
dung, 22\%) as a primary source of energy. Fuel use is reported at the
subdistrict level in 2001 and at the village level in 2011.

The Economic Censuses are complete enumerations of all nonfarm
establishments undertaken in 1990, 1998, 2005 and 2013, including
informal and non-manufacturing firms. We matched these on village
names to the three population censuses using a fuzzy matching
algorithm. The Economic Census reports total employment and industry
for all firms. We create variables describing total employment in (i)
firms engaged in logging and (ii) firms whose primary input is raw
lumber, which include sawmilling and planing of wood, manufacture of
wooden products such as furniture and wooden containers, manufacture
of cork, and manufacture of pulp and paper products. The industry
categorization for the 2005 Economic Census places logging firms in
the same industry category as firms engaged in the conservation of
forest plantations, management of forest tree nurseries and other
afforestation categories. We therefore exclude 2005 from analysis of
employment in logging firms.
